\documentclass{ltjsarticle}
\usepackage{amsmath}
\usepackage{graphicx}
\usepackage{geometry}
\usepackage{float}
\usepackage{hyperref}
\usepackage{url}
\geometry{left=25mm,right=25mm,top=25mm,bottom=25mm}

\begin{document}
\title{論理回路}
\author{平野 健汰}
\date{\today}

\maketitle

\section{目的}
ディジタルシステムを構成する論理回路について,組み合わせ回路,順序回路の動作と設計法・実装法の実習,論理素子の動作の理解.

\section{理論}
\subsection{ディジタル信号処理と論理素子}
ディジタル信号処理を行うデバイスは多岐にわたるが,その主要部分は論理回路である.
つまり,二値論理とそれを実行する論理素子からなる.
論理素子の実態は電磁リレーや真空管kバイポーラトランジスタなどの連続した入出力特性を持つスイッチング回路である.
そして,素子ごとに固有の電気的,時間的特性を持つ.
このため,高速な動作を行うディジタル信号処理回路を設計・作製するためにはこれらの特性を考慮する必要がある.

\subsection{論理素子(CMOS)の特性}
\subsubsection{動作電圧範囲(最大定格,推奨動作条件)}
ディジタルICにはIC毎に決まった動作保証に関する電気的条件がある.
これは,電源電圧,動作周波数,動作温度範囲などがある.
その条件を満足するように使う必要がある.

\noindent 絶対最大定格:端子に印加できる最大の電圧,あるいは流すことができる最大の電流.
これはICの破壊を防ぐために必ず守らなければならない.

\noindent 推奨動作条件:ICが正常に動作するための電気的条件.
これは,ICの性能を最大限に引き出すために守るべき条件である.

\subsubsection{入出力電圧特性(${V_{OH}, V_{OL}, V_{IH}, V_{IL}}$)}
ディジタルICはアナログ値を持つ入出力電圧が,高レベルか低レベルかを判定して信号処理を行う.
\begin{itemize}
  \item ${V_{OH}}$:出力ハイレベル電圧
  \item ${V_{OL}}$:出力ローレベル電圧
  \item ${V_{IH}}$:入力ハイレベル電圧
  \item ${V_{IL}}$:入力ローレベル電圧
\end{itemize}



\section{使用器具}
\begin{itemize}
  \item ブレッドボード
  \item LED
  \item 抵抗
  \item 論理ICチップ
  \item ジャンパーワイヤー
\end{itemize}

\section{実験}
\subsection{実験方法}
\subsubsection{リングオシレータ}
\begin{itemize}
  \item ブレッドボード上にインバータ回路をロジックIC(インバータ)を用いて実装する.
  \item 2種類のコンデンサ(470 nF, 4.7 $\mu$F)を用いてファンクションジェネレータより方形波(f = 100 Hz, Vpp = 5 V, offset = 2.5 V)を入力する.
  \item 出力波形をオシロスコープで観測する.
\end{itemize}

\begin{itemize}
  \item ブレッドボード上に三段の縦続接続インバータによるリングオシレータ回路を実装する.
  \item V1, V2 を観測して1素子の伝搬遅延時間,発振周波数を求める.
  \item 各段のコンデンサを変更して同様に測定する.
\end{itemize}

\subsubsection{Dフリップフロップを用いた2ビット4進非同期カウンタ回路}
\begin{itemize}
  \item ブレッドボード上に2ビットカウンタ回路をロジックIC(エッジトリガ型Dフリップフロップ)を用いて実装.
  \item ファンクションジェネレータ(f = 1 kHz, Vpp = 5 V, offset = 2.5 V の方形波)をクロック信号CLKとして用いて駆動させる.
  \item Q1, Q2 およびクロック CLK を観測して記録した.
\end{itemize}

\subsubsection{2ビットデコーダ回路によるLEDルーレット}
\begin{itemize}
  \item ブレッドボード上に2ビットデコーダ回路と LED 回路をロジックIC(2入力論理和)を用いて実装する.
  \item 2bit4 進カウンタ回路と接続し,FG(f = 10 Hz, Vpp = 5 V, offset = 2.5 V の方形波)をクロック信号CLKとして用いて駆動させる.
  \item LED の点灯状態を観測して記録する.
\end{itemize}

\begin{itemize}
  \item リングオシレータと2ビット4 進非同期カウンタ回路を接続する.
  \item 間にはタクトスイッチを挟む
  \item たくとスイッチが推されている間,LED がルーレットのように点滅し,スイッチをオフにすると4つのLEDのうち1つのLEDのみが点灯することを確認し,記録する.
\end{itemize}
\subsection{結果}
% 実験結果を記述

\section{考察}
% 実験結果の考察を記述

\begin{thebibliography}{9}
\bibitem{reference1} \url{https://example.com/reference1}
\end{thebibliography}

\end{document}
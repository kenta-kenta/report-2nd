\documentclass{ltjsarticle}
\usepackage{amsmath}
\usepackage{graphicx}
\usepackage{geometry}
\usepackage{float}
\usepackage{hyperref}
\usepackage{url}
\geometry{left=25mm,right=25mm,top=25mm,bottom=25mm}

\begin{document}
\title{論理回路}
\author{平野 健汰}
\date{\today}

\maketitle

\section{目的}
論理回路の基本的な動作を理解し、組み合わせ回路の設計方法を習得する。

\section{理論}
% 理論部分は実際の実験内容に応じて記述

\section{使用器具}
\begin{itemize}
  \item ブレッドボード
  \item LED
  \item 抵抗
  \item 論理ICチップ
  \item ジャンパーワイヤー
\end{itemize}

\section{実験}
\subsection{実験方法}
\subsubsection{リングオシレータ}
\begin{itemize}
  \item ブレッドボード上にインバータ回路をロジックIC(インバータ)を用いて実装する.
  \item 2種類のコンデンサ(470 nF, 4.7 $\mu$F)を用いてファンクションジェネレータより方形波(f = 100 Hz, Vpp = 5 V, offset = 2.5 V)を入力する.
  \item 出力波形をオシロスコープで観測する.
\end{itemize}

\begin{itemize}
  \item ブレッドボード上に三段の縦続接続インバータによるリングオシレータ回路を実装する.
  \item V1, V2 を観測して1素子の伝搬遅延時間,発振周波数を求める.
  \item 各段のコンデンサを変更して同様に測定する.
\end{itemize}

\subsubsection{Dフリップフロップを用いた2ビット4進非同期カウンタ回路}
\begin{itemize}
  \item ブレッドボード上に2ビットカウンタ回路をロジックIC(エッジトリガ型Dフリップフロップ)を用いて実装.
  \item ファンクションジェネレータ(f = 1 kHz, Vpp = 5 V, offset = 2.5 V の方形波)をクロック信号CLKとして用いて駆動させる.
  \item Q1, Q2 およびクロック CLK を観測して記録した.
\end{itemize}

\subsubsection{2ビットデコーダ回路によるLEDルーレット}
\begin{itemize}
  \item ブレッドボード上に2ビットデコーダ回路と LED 回路をロジックIC(2入力論理和)を用いて実装する.
  \item 2bit4 進カウンタ回路と接続し,FG(f = 10 Hz, Vpp = 5 V, offset = 2.5 V の方形波)をクロック信号CLKとして用いて駆動させる.
  \item LED の点灯状態を観測して記録する.
\end{itemize}

\begin{itemize}
  \item リングオシレータと2ビット4 進非同期カウンタ回路を接続する.
  \item 間にはタクトスイッチを挟む
  \item たくとスイッチが推されている間,LED がルーレットのように点滅し,スイッチをオフにすると4つのLEDのうち1つのLEDのみが点灯することを確認し,記録する.
\end{itemize}
\subsection{結果}
% 実験結果を記述

\section{考察}
% 実験結果の考察を記述

\begin{thebibliography}{9}
\bibitem{reference1} \url{https://example.com/reference1}
\end{thebibliography}

\end{document}
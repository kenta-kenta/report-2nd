\documentclass{ltjsarticle}
\usepackage{listings}


\begin{document}

\title{データ構造とアルゴリズム演習}
\author{1TE23035E 平野 健汰}
\date{\today}

\maketitle
はじめに今回作成したコードはgithubにアップロードしています.以下のリンクからアクセスできます.
\begin{verbatim}
  https://github.com/kenta-kenta/clang-algorithm
\end{verbatim}

\part{Sort}

\section{Merge Sort}
\subsection{概略}
数値が格納された配列を分割し,それぞれをソートした後,マージすることでソートを行うアルゴリズムである.
\subsection{仕様}
\noindent 入力:整数が格納された配列,最初の要素のインデックス,最後の要素のインデックス

\noindent 出力:整数が格納された配列:for文で配列の要素を出力することでソートされた配列を出力する.
\subsection{コード}

\begin{lstlisting}[frame=single, lineskip=-5pt]
#include <stdio.h>

void mergeSort(int arr[], int left, int right)
{
    int i, j, k, mid, tmp[10000];

    if (left < right)
    {
        mid = (left + right) / 2;
        mergeSort(arr, left, mid);
        mergeSort(arr, mid + 1, right);

        for (i = left; i <= mid; i++)
        {
            tmp[i] = arr[i];
        }

        for (j = mid + 1; j <= right; j++)
        {
            tmp[right - (j - mid - 1)] = arr[j];
        }

        i = left;
        j = right;

        for (k = left; k <= right; k++)
        {
            if (tmp[i] < tmp[j])
            {
                arr[k] = tmp[i];
                i++;
            }
            else
            {
                arr[k] = tmp[j];
                j--;
            }
        }
    }
}

void main()
{
    int arr[] = {3, 1, 4, 9, 2, 6, 5, 7, 8, 10};
    mergeSort(arr, 0, 9);
    for (int i = 0; i < 11; i++)
    {
        printf("%d\n", arr[i]);
    }
}
\end{lstlisting}

\subsection{プログラムの特徴}
マージソートは配列を分割し,それぞれをソートした後,マージすることでソートを行うアルゴリズムである.

まず,配列を2つに分割し,再帰的にマージソートを呼び出しソートする.次にソートされた部分配列を,左側の部分配列はそのままコピーし,右側の部分配列は逆順にコピーする.最後に,左側の部分配列と右側の部分配列を比較し,小さい方を配列に格納する.

このプログラムの計算量は$O(n\log n)$である.
\subsection{まとめ}
マージソートはそこまで実装が難しくなかったが,今回作成したプログラムは一時配列の大きさが固定されているので,配列の要素数が多い場合にはエラーが発生する可能性がある.
再帰的な関数であったので,再帰の理解が深まった.また,再帰的な関数を用いることで,プログラムの見通しがよくなった.

\section{Quick Sort}
\subsection{概略}
数値が格納された配列をピボットを基準に分割し,それぞれをソートすることでソートを行うアルゴリズムである.

\subsection{仕様}
\noindent 入力:整数が格納された配列,最初の要素のインデックス,最後の要素のインデックス

\noindent 出力:整数が格納された配列:for文で配列の要素を出力することでソートされた配列を出力する.

\subsection{コード}

\begin{lstlisting}[frame=single, lineskip=-5pt]
#include <stdio.h>
int partision(int arr[], int start, int end)
{
    int pivot = arr[end];
    int i = start - 1;
    int keep, j;
    for (j = start; j < end; j++)
    {
        if (arr[j] <= pivot)
        {
            i = i + 1;
            keep = arr[i];
            arr[i] = arr[j];
            arr[j] = keep;
        }
    }
    keep = arr[i + 1];
    arr[i + 1] = arr[j];
    arr[j] = keep;
    return i + 1;
}

void quickSort(int arr[], int start, int end)
{
    if (start < end)
    {
        int q = partision(arr, start, end);
        quickSort(arr, start, q - 1);
        quickSort(arr, q + 1, end);
    }
}

void main()
{
    int arr[] = {3, 1, 4, 9, 2, 6, 5, 7, 8, 10};
    quickSort(arr, 0, 9);
    for (int i = 0; i < 11; i++)
    {
        printf("%d\n", arr[i]);
    }
}
\end{lstlisting}

\subsection{プログラムの特徴}
クイックソートは配列をピボットを基準に分割し,それぞれをソートすることでソートを行うアルゴリズムである.

このプログラムはpartision関数と,quickSort関数の2つの関数で構成されている.partision関数は配列を基準値より小さい値と大きい値に分割する関数である.
partision関数は,配列の最後の要素をピボットとして選択し,それを基準にピボットより小さい値と大きい値配列を分割する.
quickSort関数は配列を分割し,それぞれをソートする関数である.partision関数で分割された配列を再帰的にquickSort関数を呼び出し,ソートする.

このプログラムの計算量は最悪の場合$O(n^2)$であるが,平均計算量は$O(n\log n)$である.
\subsection{まとめ}
クイックソートは再帰的な関数を用いることで,プログラムの見通しがよくなった.また,配列を分割することで,ソートを行うことができることがわかった.
クイックソートは実装はそこまで難しくなかった.展望としては,今回は再帰関数を利用したが,末尾再帰を用いることで,スタックオーバーフローを防ぐことができるので,今後実装してみたい.

\section{Randomized Quick Sort}
\subsection{概略}
数値が格納された配列をランダムにピボットを選択し,それを基準に分割し,それぞれをソートすることでソートを行うアルゴリズムである.

\subsection{仕様}
\noindent 入力:整数が格納された配列,最初の要素のインデックス,最後の要素のインデックス

\noindent 出力:整数が格納された配列:for文で配列の要素を出力することでソートされた配列を出力する.
\subsection{コード}

\begin{lstlisting}[frame=single, lineskip=-5pt]
#include <stdio.h>
int partision(int arr[], int start, int end)
{
    int pivot = arr[end];
    int i = start - 1;
    int keep, j;
    for (j = start; j < end; j++)
    {
        if (arr[j] <= pivot)
        {
            i = i + 1;
            keep = arr[i];
            arr[i] = arr[j];
            arr[j] = keep;
        }
    }
    keep = arr[i + 1];
    arr[i + 1] = arr[j];
    arr[j] = keep;
    return i + 1;
}

int RandomPartision(int arr[], int start, int end)
{
    int i = rand() % (end - start + 1) + start;
    int keep = arr[i];
    arr[i] = arr[end];
    arr[end] = keep;
    return partision(arr, start, end);
}

void RandomQuickSort(int arr[], int start, int end)
{
    if (start < end)
    {
        int q = RandomPartision(arr, start, end);
        RandomQuickSort(arr, start, q - 1);
        RandomQuickSort(arr, q + 1, end);
    }
}

void main()
{
    int arr[] = {3, 1, 4, 9, 2, 6, 5, 7, 8, 10};
    RandomQuickSort(arr, 0, 9);
    for (int i = 0; i < 11; i++)
    {
        printf("%d\n", arr[i]);
    }
}
\end{lstlisting}

\subsection{プログラムの特徴}
ランダムクイックソートはクイックソートのピボットをランダムに選択することで,最悪の場合の計算量を防ぐアルゴリズムである.
大まかな流れはクイックソートと同じであるので省略する.
違うところは,RandomPartision関数でランダムにピボットを選択し,それを基準に分割する.

このプログラムの計算量は最悪の場合$O(n^2)$であるが,平均計算量は$O(n\log n)$である.
ただし,ランダムクイックソートはクイックソートよりも安定している.

\subsection{まとめ}
このプログラムの実装はクイックソートのピボットをランダムにしただけであったので比較的容易だった.
クイックソートと比較して,ランダムクイックソートは安定していることがわかった.
安定しているソートを利用することで,プログラムの信頼性が向上するので,今後もランダムクイックソートを利用していきたい.

\section{Radix Sort}
\subsection{概略}
数値が格納された配列を桁ごとにソートすることで全体のソートを行う基数ソートと呼ばれるアルゴリズムである.

\subsection{仕様}
\noindent 入力:数値が格納された配列,配列の要素数

\noindent 出力:整数が格納された配列:for文で配列の要素を出力することでソートされた配列を出力する.
\subsection{コード}

\begin{lstlisting}[frame=single, lineskip=-5pt]
#include <stdio.h>
void RadixSort(int arr[], int n)
{
    int i, j, k, m, p = 1, index = 0;
    int tmp[10][10000];
    for (i = 0; i < n; i++)
    {
        if (arr[i] > index)
        {
            index = arr[i];
        }
    }

    int maxDigits = 0;
    while (index > 0)
    {
        maxDigits++;
        index /= 10;
    }

    for (int digit = 0; digit < maxDigits; digit++)
    {
        for (i = 0; i < 10; i++)
        {
            for (j = 0; j < 10000; j++)
            {
                tmp[i][j] = -1;
            }
        }

        for (i = 0; i < n; i++)
        {
            index = (arr[i] / p) % 10;
            for (j = 0; j < 10000; j++)
            {
                if (tmp[index][j] == -1)
                {
                    tmp[index][j] = arr[i];
                    break;
                }
            }
        }

        m = 0;
        for (i = 0; i < 10; i++)
        {
            for (j = 0; j < 10000; j++)
            {
                if (tmp[i][j] != -1)
                {
                    arr[m] = tmp[i][j];
                    m++;
                }
            }
        }
        p *= 10;
    }
}

void main()
{
    int arr[] = {3, 1, 4, 9, 2, 6, 5, 7, 8, 10};
    RadixSort(arr, 10);
    for (int i = 0; i < 11; i++)
    {
        printf("%d\n", arr[i]);
    }
}
\end{lstlisting}

\subsection{プログラムの特徴}
Radix Sortは数値が格納された配列を桁ごとにソートすることでソートを行うアルゴリズムである.
このプログラムは,数値の桁数を求め,それを基準にソートを行う.

このプログラムは,まず,配列の要素数を求め,それを基準に10$\times $10000の2次元配列を作成する.(ここで10000は配列の要素数が10000を超えないとして決めたものである.)
入力された配列の要素をその桁の値に応じて2次元配列に格納する.
すべての要素が格納されたら,2次元配列の要素を1次元の配列にソートして格納する.
この操作を桁数分繰り返すことで,ソートを行う.

このプログラムの計算量は,桁数を$k$としたとき$O(n\times k)$である.

\subsection{まとめ}
今回は,2次元配列を利用してソートを行ったが,2次元配列を利用することで,桁ごとにソートを行うことができた.
ただ,基数ソートの実装は2次元配列を用いない方法もあるので,今後はその方法も実装してみたい.
また,基数ソートはほかのソートと組み合わせることもできそうだと感じたので,組み合わせて実装してみたい.


\section{ソートの比較}
\subsection{概略}
実装したソート (挿入ソート,ヒープソート,マージソート,クイックソート) の計算時間を入
力サイズを変えて比較する.

\subsection{結果}
表1にソートアルゴリズムの実行時間を示す.
5回実行し,その平均を取った.
\begin{table}[htbp]
    \centering
    \caption{表1:ソートアルゴリズムの実行時間比較(s)}
    \begin{tabular}{|c|c|c|c|c|}
        \hline
        \textbf{入力サイズ} & \textbf{挿入ソート} & \textbf{ヒープソート} & \textbf{マージソート} & \textbf{クイックソート} \\
        \hline
        1000 & 0.001626 & 0.000217 & 0.005167 & 0.000171 \\
        10000 & 0.143252 & 0.001538 & 0.029646 & 0.001261 \\
        100000 & 14.048372 & 0.021621 & 0.270004 & 0.012222 \\
        \hline
    \end{tabular}
\end{table}


\part{Order Statistics}
\section{最大値及び最小値}
\subsection{概略}
数値が格納された配列の最大値及び最小値を求めるアルゴリズムである.
\subsection{仕様}
\noindent 入力:整数が格納された配列,配列の要素数

\noindent 出力:整数:最大値及び最小値を出力する.
\subsection{コード}

\begin{lstlisting}[frame=single, lineskip=-5pt]
#include <stdio.h>
int Max(int arr[], int n)
{
    int i, max = arr[0];
    for (i = 1; i < n; i++)
    {
        if (arr[i] > max)
        {
            max = arr[i];
        }
    }
    return max;
}

int Min(int arr[], int n)
{
    int i, min = arr[0];
    for (i = 1; i < n; i++)
    {
        if (arr[i] < min)
        {
            min = arr[i];
        }
    }
    return min;
}
void main()
{
    int arr[10] = {1, 3, 5, 7, 9, 2, 4, 6, 8, 10};
    printf("%d\n", Max(arr, 10));
    printf("%d\n", Min(arr, 10));
}
\end{lstlisting}

\subsection{プログラムの特徴}
このプログラムは数値が格納された配列から最大値及び最小値を求めるアルゴリズムである.
このプログラムはfor文を用いて,配列の要素を比較し,最大値及び最小値を求める.
最大値,最小値を求めるだけであるので,ソートされているわけではない.

このプログラムの計算量は$O(n)$である.

\subsection{まとめ}
このプログラムは最大値,最小値を比較するだけでそこまで難しくなかった.
最大値,最小値を求めるアルゴリズムの計算量は$O(n)$であり,より高速にすることはできないが,ソート済みの配列だったり,二分探索木を利用することで,より高速に値を求めることはできる.

\section{中央値}
\subsection{概略}
数値が格納された配列の中央値を求めるアルゴリズムである.
\subsection{仕様}
\noindent 入力:整数が格納された配列,配列の要素数

\noindent 出力:整数:中央値を出力する.
\subsection{コード}
\begin{lstlisting}[frame=single, lineskip=-5pt]
#include <stdio.h>
int partision(int arr[], int start, int end)
{
    int pivot = arr[end];
    int i = start - 1;
    int keep, j;
    for (j = start; j < end; j++)
    {
        if (arr[j] <= pivot)
        {
            i = i + 1;
            keep = arr[i];
            arr[i] = arr[j];
            arr[j] = keep;
        }
    }
    keep = arr[i + 1];
    arr[i + 1] = arr[j];
    arr[j] = keep;
    return i + 1;
}

int RandomPartision(int arr[], int start, int end)
{
    int i = rand() % (end - start) + start;
    int keep = arr[i];
    arr[i] = arr[end];
    arr[end] = keep;
    return partision(arr, start, end);
}

int RandomizedSelect(int arr[], int pivot, int end, int select)
{
    if (pivot == end)
    {
        return arr[pivot];
    }
    int q = RandomPartision(arr, pivot, end);
    int k = q - pivot + 1;
    if (select == k)
    {
        return arr[q];
    }
    else if (select < k)
    {
        return RandomizedSelect(arr, pivot, q - 1, select);
    }
    else
    {
        return RandomizedSelect(arr, q + 1, end, select - k);
    }
}

void main()
{
    int arr[10] = {1, 3, 5, 7, 9, 2, 4, 6, 8, 10};
    printf("%d\n", RandomizedSelect(arr, 0, 9, 6));
}
\end{lstlisting}

\subsection{プログラムの特徴}
このプログラムは中央値を求めるアルゴリズムである.
このプログラムはランダム化された選択アルゴリズムを用いて中央値を求める.
ランダムにピボットを選択し,それを基準に分割することで中央値を求める.

このプログラムの期待実行時間は$O(n)$である.
\subsection{まとめ}
はじめは中央値を求めるプログラムは最大値や最小値を求めるプログラムと比べて難しそうで実行時間も長そうだと思っていたが,
教科書を読んでみると,ランダム化された選択アルゴリズムを用いることで,期待実行時間は$O(n)$であることがわかった.
展望としては,今回はランダム化された選択アルゴリズムを用いたが,他の選択アルゴリズムを用いて中央値を求めることもできるので,今後実装してみたい.

\section{n/3番目に大きい数値}
\subsection{概略}
数値が格納された配列のn/3番目に大きい数値を求めるアルゴリズムである.
\subsection{仕様}
\noindent 入力:整数が格納された配列,配列の要素数

\noindent 出力:整数:n/3番目に大きい数値を出力する.
\subsection{コード}
\begin{lstlisting}[frame=single, lineskip=-5pt]
#include <stdio.h>
int partision(int arr[], int start, int end)
{
    int pivot = arr[end];
    int i = start - 1;
    int keep, j;
    for (j = start; j < end; j++)
    {
        if (arr[j] <= pivot)
        {
            i = i + 1;
            keep = arr[i];
            arr[i] = arr[j];
            arr[j] = keep;
        }
    }
    keep = arr[i + 1];
    arr[i + 1] = arr[j];
    arr[j] = keep;
    return i + 1;
}

int RandomPartision(int arr[], int start, int end)
{
    int i = rand() % (end - start) + start;
    int keep = arr[i];
    arr[i] = arr[end];
    arr[end] = keep;
    return partision(arr, start, end);
}

int RandomizedSelect(int arr[], int pivot, int end, int select)
{
    if (pivot == end)
    {
        return arr[pivot];
    }
    int q = RandomPartision(arr, pivot, end);
    int k = q - pivot + 1;
    if (select == k)
    {
        return arr[q];
    }
    else if (select < k)
    {
        return RandomizedSelect(arr, pivot, q - 1, select);
    }
    else
    {
        return RandomizedSelect(arr, q + 1, end, select - k);
    }
}

int Selection(int arr[], int n, int select)
{
    return RandomizedSelect(arr, 0, n - 1, n - select + 1);
}

void main()
{
    int arr[10] = {1, 3, 5, 7, 9, 2, 4, 6, 8, 10};
    printf("%d\n", Selection(arr, 10, 3));
}
\end{lstlisting}
\subsection{プログラムの特徴}
このプログラムは,前問で作成したランダム化された選択アルゴリズムを用いて,n/3番目に大きい数値を求めるアルゴリズムである.
具体的には,n - select + 1番目に大きい数値を求めることで,n/3番目に大きい数値を求める.
つまり,どのk番目の値であっても求めることができるが,main関数の中で,3/n番目に大きい値を求めることを指定している.

このプログラムの期待実行時間は$O(n)$である.

\subsection{まとめ}
今回作成したのは選択アルゴリズムと呼ばれるアルゴリズムであった.
作成したのは前問の際に作ったものであった.

\section{参考文献}
[1] T.コルメン・C.ライザーソン・他,アルゴリズムとイントロダクション 第 4 版,
p132~p207,2023

\end{document}
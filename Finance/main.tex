\documentclass[a4paper]{ltjsarticle}
\usepackage{geometry}
\geometry{left=25mm,right=25mm,top=25mm,bottom=25mm}
\usepackage{graphicx}
\usepackage{amsmath}
\usepackage{float}
\usepackage{url}

% 1ページあたり40文字×40行の設定
\renewcommand{\baselinestretch}{1.0}
\setlength{\topskip}{0mm}
\setlength{\parskip}{0mm}

\begin{document}

\title{Schoo株式会社の財務分析と企業価値評価}
\author{1TE23035E 平野健汰}
\date{\today}

\maketitle

\section{はじめに}
株式会社Schooを対象に,財務諸表分析と企業価値評価を行う.
財務諸表分析は同時期に上場したタイミーと比較して財務指標と戦略について評価する.
企業価値評価はDCFモデルとDDモデルを用いて株価を算定し,公募価格,初値について評価する.

\section{企業概要}
株式会社Schooは「世の中から卒業をなくす」をミッションに,オンラインでの社会人教育に取り組んできた.
株式会社Schooの事業内容は,インターネットでの学びや教育を起点とした社会変革.
2011年に設立され,2024年に東京証券取引所グロース市場に上場した.

\section{財務諸表分析}
\subsection{ROE分析}
ROEとは,Return on Equityの略で,株主資本に対する当期純利益の比率を計算することによって「経営の効率性」や「株主に対する還元効率」を示す財務指標のことである.
ROEは次の式で表すことができる.
\[
ROE = \frac{当期純利益}{自己資本}
\]

ROEは,次の3つの要素に分解することができる.
\begin{align*}
ROE &= 売上高利益率 \times 総資本回転率 \times 財務レバレッジ率 \\
ROE &= \frac{当期純利益}{売上高} \times \frac{売上高}{総資本} \times \frac{総資本}{自己資本}
\end{align*}

2022年10月から2023年9月までのROEを計算すると以下のように求められた.

\[
ROE = -\frac{680,386}{165,499} = -4.110
\]

\[
売上高利益率 = -\frac{680,386}{2,007,927} = -0.338
\]

\[
総資本回転率 = \frac{2,007,927}{2,098,296} = 0.957
\]

\[
財務レバレッジ率 = \frac{2,098,296}{165,499} = 12.682
\]

これをもとに2023年10月から2024年9月までのROEを計算する.

\[
ROE = \frac{184,387}{694,879} = 0.265
\]

\[
売上高利益率 = \frac{184,387}{2,852,780} = 0.065
\]

\[
総資本回転率 = \frac{2,852,780}{2,415,319} = 1.181
\]

\[
財務レバレッジ率 = \frac{2,415,319}{694,879} = 3.475
\]

以上の結果から,ROEは2023年10月から2024年9月までに大幅に改善されたことがわかる.
財務レバレッジ率が大幅に減少し,売上高利益率,総資本回転率が増加したことから,初期費用が大きかったが,その後の売上高が増加し,収益性が向上したと考えられる.

\subsection{タイミーとの比較分析}
株式会社タイミーは,「一人ひとりの時間を豊かに」をミッションに,働きたい時間と働いてほしい時間をマッチングする「タイミー」というサービスと,地方で「はたらく」体験を通じて第二の故郷を見つけられる「タイミートラベル」というサービスを展開している.
2017年に設立され,2024年に東京証券取引所グロース市場に上場した.

まず,タイミーのROEを計算する.
2021年11月から2022年10月までのROEを計算すると以下のように求められた.

\[
ROE = \frac{256,751}{4,379,895} = 0.059
\]

\[
売上高利益率 = \frac{256,751}{6,216,517} = 0.041
\]

\[
総資本回転率 = \frac{6,216,517}{4,390,717} = 1.415
\]

\[
財務レバレッジ率 = \frac{4,390,717}{4,379,895} = 1.002
\]

2022年11月から2023年10月までのROEを計算すると以下のように求められた.

\[
ROE = \frac{1,802,769}{6,201,964} = 0.290
\]

\[
売上高利益率 = \frac{1,802,769}{16,144,584} = 0.112
\]

\[
総資本回転率 = \frac{16,144,584}{16,830,027} = 0.959
\]

\[
財務レバレッジ率 = \frac{16,830,027}{6,201,964} = 2.713
\]

SchooのROE分解とタイミーのROE分解を比較すると,以下のようになった.
\begin{table}[H]
  \begin{center}
    \caption{ROE分解比較}
    \begin{tabular}{|c|c|c|c|c|c|c|} \hline
      & ROE & 売上高利益率 & 総資本回転率 & 財務レバレッジ率 \\ \hline
      Schoo & 0.265 & 0.065 & 1.181 & 3.475 \\ \hline
      タイミー & 0.290 & 0.112 & 0.959 & 2.713 \\ \hline
    \end{tabular}
  \end{center}
\end{table}

Schoo株式会社は,売上高利益率が低く,総資本回転率が高く,財務レバレッジ率が高い.
一方,タイミーは,売上高利益率が高く,総資本回転率が低く,財務レバレッジ率が低い.

\subsection{戦略評価}
Schooは,前節の分析から,積極的に投資を行い,効率的に売り上げをあげているものの,利益を十分に確保できていないことが考えられる.
決算説明資料より,事業戦略として「社会人教育の第一想起」を獲得することを目指し,国内社会人教育市場のNo.1を目指すとしている.
また,リカーリング収益モデルを採用しており,これは契約の継続の維持・向上が重要である.

一方,タイミーは,前節の分析から,高い利益率を確保できているものの,資産を効率的に活用できていなく,自己資本中心の経営を行っていると考えられる.
決算説明資料より,事業戦略としてサービスの拡充・開発を継続的に行い,ワーカーとクライアント双方のニーズに対応することを目指している.

以上の内容をもとにこれら2社の戦略の比較をしてみると,Schooは成長を重視した戦略を取っていると考えられる.
積極的に投資を行い,顧客基盤を拡大することを重視していると考えられる.

一方,タイミーは,収益性を重視した戦略を取っていると考えられる.
今後は,アライアンスやサービス拡充により事業の多角化を図ることでさらなる成長を目指すと考えられる.

\section{企業価値評価}
\subsection{DCFモデルによる評価}
DCFモデルで企業価値評価を行う.
DCFモデルは,企業の将来のフリーキャッシュフロー(FCF)を割引して現在価値を求めるモデルである.
モデルを利用するにあたって,次の仮説を設定する.
株主資本コストは8\%とする.
法人税は30\%とする.
FCFの成長率は1年目から2年目まで40\%, 3年目から5年目まで30\%, 6年目以降は5\%とする.


\[
DCFモデルの企業価値 = \frac{FCF_1}{1 + r} + \frac{FCF_2}{(1 + r)^2} + \cdots + \frac{FCF_n}{(1 + r)^n}
= \sum_{t=1}^{n} \frac{FCF_t}{(1 + r)^t}
\]

\[
DCFモデルの企業価値 = \sum_{t=1}^{5} {FCFの現在価値} + \frac{6年目のFCF}{WACC}
\]

\[
割引率(WACC) r = \frac{自己資本比率 \times 株主資本コスト}{負債比率 \times 負債コスト \times (1 - 法人税)}
\]

\begin{table}[H]
  \begin{center}
    \caption{DCFモデルによる企業価値評価}
    \begin{tabular}{|c|c|c|c|c|c|c|c|} \hline
      年度 & 2024(当期) & 2025 & 2026 & 2027 & 2028 & 2029 & その後の成長 \\ \hline
      営業CF & 115,178 &  &  &  &  &  & \\ \hline
      投資CF & 19,806 &  &  &  &  &  & \\ \hline
      FCF & 95,372 & 133,520 & 186,929 & 243,007 & 315,910 & 410,683 & 431,217 \\ \hline
      FCFの成長率 &  & 40.0\% & 40.0\% & 30.0\% & 30.0\% & 30.0\% & 5.0\% \\ \hline
      株主資本コスト & 8${\%}$ & 8${\%}$ & 8${\%}$ & 8${\%}$ & 8${\%}$ & 8\% & 8\% \\ \hline
      負債コスト & 2.78\% & 2.78\% & 2.78\% & 2.78\% & 2.78\% & 2.78\% & 2.78\% \\ \hline
      自己資本比率 & 26.91\% & 26.91\% & 26.91\% & 26.91\% & 26.91\% & 26.91\% & 26.91\% \\ \hline
      WACC &  & 3.58 & 3.58 & 3.58 & 3.58 & 3.58 & 3.58 \\ \hline
      FCFの現在価値 &  & 128,905 & 172,230 & 218,670 & 274,447 & 361,673 & 12,045,167 \\ \hline
    \end{tabular}
  \end{center}
\end{table}

\begin{table}
  \begin{center}
    \caption{企業価値評価}
    \begin{tabular}{|c|c|} \hline
      FCVのPV合計 & 13,201,092 \\ \hline
      純資産(時価)& 11,435,652 \\ \hline
      発行済株数 & 12,422,700 \\ \hline
      株価 & 920.54 \\ \hline
    \end{tabular}
  \end{center}
\end{table}


\subsection{DDモデルによる評価}
DDモデルで企業価値評価を行う.
モデルを利用するにあたって,以下の仮説を設定する.
売上の成長率は1年目から2年目まで40\%, 3年目から5年目まで30\%, 6年目以降は5\%とする.
配当性向は30\%とする.
株主資本コストは8\%とする.

\[
DD法での株式価値 = \sum_{t=1}^{5} {t年目の配当額} + \frac{6年目の配当額}{株主資本コスト}
\]

\begin{table}[H]
  \begin{center}
    \caption{DDモデルによる株価算定}
    \begin{tabular}{|c|c|c|c|c|c|c|c|} \hline
      年度 & 2024(当期) & 2025 & 2026 & 2027 & 2028 & 2029 & その後の成長 \\ \hline
      売上高 & 2,852,780 & 3,993,892 & 5,591,448 & 7,268,883 & 9,449,548 & 12,284,413 & 12,898,633 \\ \hline
      売上の成長率 &  & 40.0\% & 40.0\% & 30.0\% & 30.0\% & 30.0\% & 5.0\% \\ \hline
      税引後利益 & 184,387 & 258,005 & 361,207 & 469,569 & 610,440 & 793,573 & 833,251 \\ \hline
      売上当期利益率 & 6.46\% & 6.46\% & 6.46\% & 6.46\% & 6.46\% & 6.46\% & 6.46\% \\ \hline
      配当性向 &  & 30\% & 30\% & 30\% & 30\% & 30\% & 30\% \\ \hline
      配当額 &  & 77,401 & 108,362 & 140,870 & 183,131 & 238,071 & 249,975 \\ \hline
      株主資本コスト &  & 8\% & 8\% & 8\% & 8\% & 8\% & 8\% \\ \hline
      配当の現在価値 &  & 71,208 & 99,692 & 129,600 & 168,480 & 219,024 & 3,124,687 \\ \hline
    \end{tabular}
  \end{center}
\end{table}

\begin{table}[H]
  \begin{center}
    \caption{DDモデルによる株価算定}
    \begin{tabular}{|c|c|} \hline
      配当のPV合計 & 3,815,691 \\ \hline
      発行済株数 & 12,422,700 \\ \hline
      株価 & 307.00 \\ \hline
    \end{tabular}
  \end{center}
\end{table}
\subsection{公開価格と初値の評価}
まず,DCFモデル,DDモデルの計算に利用した仮説で結果に大きく影響すると考えられるものを以下に示す.
\begin{itemize}
  \item FCFの成長率
  \item 株主資本コスト
  \item 配当性向
\end{itemize}

成長率は,企業の将来の成長性を示す指標であり,これが高いほど企業価値が高くなる.
成長率は時間経過とともに低下すると考えられる.
また,成長率は業界の動向や企業の戦略によって変化する.
競合他社を考えてみると,UdemyやCourseraなどが挙げられる.
ただ,これらは海外企業であり,日本国内での競合他社は少ないと考える.
以上の内容より,設定した仮説は妥当であると考えられる.

株主資本コストは,株式での資金調達に必要なコストであり,これが高いほど企業価値が低くなる.
今回は,8\%と設定したが,これは一般的な株主資本コストである.
ただ,株主資本コストは企業の信用度や市場の状況によって変化する.
また,株主資本コストは,企業の業績や将来の成長性によっても変化する.
以上の内容より,設定した仮説は妥当であると考えられる.

配当性向は,企業の利益を株主に還元する割合を示す指標である.
今回は,30\%と設定したが,これは一般的な配当性向である.
ただ,配当性向は企業の業績や将来の成長性によって変化する.
以上の内容より,設定した仮説は妥当であると考えられる.

以上の内容から,DCFモデル,DDモデルの計算に利用した仮説は妥当であると考えられる.

株価の評価を行う.
DCFモデルによる株価は920.54円,DDモデルによる株価は307.00円であった.
公募価格は690円であり,初値は761円であった.
DCFモデルによる株価は公募価格よりも高く,初値よりも高い.
DDモデルによる株価は公募価格よりも低く,初値よりも低い.
以上の結果から,DCFモデルによる株価は過大評価であり,DDモデルによる株価は過小評価であると考えられる.
今回,DDモデルの結果が市場評価と大きく乖離している.
これは,DDモデルの仮説が妥当でないことが考えられる.
具体的には,売上成長率が低い,配当性向が低いことが考えられる.

\begin{thebibliography}{9}
\bibitem{Schoo_SecuritiesReport} Schooの有価証券報告書 \url{https://ssl4.eir-parts.net/doc/264A/yuho_pdf/S100V0PZ/00.pdf}
\bibitem{Schoo_FinancialResultsBriefingReport} Schooの決算説明資料 \url{https://ssl4.eir-parts.net/doc/264A/ir_material_for_fiscal_ym/167570/00.pdf}
\bibitem{Timee_SecuritiesReport} Timeeの有価証券報告書 \url{https://corp.timee.co.jp/ir/securities/}
\bibitem{Timee_FinancialResultsBriefingReport} Timeeの決算説明資料 \url{https://corp.timee.co.jp/ir/presentations/}
\end{thebibliography}

\end{document}
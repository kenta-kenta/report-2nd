\documentclass[a4paper]{ltjsarticle}
\usepackage{geometry}
\geometry{left=25mm,right=25mm,top=25mm,bottom=25mm}
\usepackage{graphicx}
\usepackage{amsmath}
\usepackage{float}


% 1ページあたり40文字×40行の設定
\renewcommand{\baselinestretch}{1.0}
\setlength{\topskip}{0mm}
\setlength{\parskip}{0mm}

\begin{document}

\title{Schoo株式会社の財務分析と企業価値評価}
\author{平野健汰}
\date{\today}

\maketitle

\section{はじめに}
% 分析の目的と概要

\section{企業概要}
株式会社Schooは「世の中から卒業をなくす」をミッションに,オンラインでの社会人教育に取り組んできた.
株式会社Schooの事業内容は,インターネットでの学びや教育を起点とした社会変革.
2011年に設立され,2024年に東京証券取引所グロース市場に上場した.

\section{財務諸表分析}
\subsection{ROE分析}
ROEとは,Return on Equityの略で,株主資本に対する当期純利益の比率を計算することによって「経営の効率性」や「株主に対する還元効率」を示す財務指標のことである.
ROEは次の式で表すことができる.
\[
ROE = \frac{当期純利益}{自己資本}
\]

ROEは,次の3つの要素に分解することができる.
\begin{align*}
ROE &= 売上高利益率 \times 総資本回転率 \times 財務レバレッジ率 \\
ROE &= \frac{当期純利益}{売上高} \times \frac{売上高}{総資本} \times \frac{総資本}{自己資本}
\end{align*}

これをもとに2023年10月から2024年9月までのROEを計算する.

2022年10月から2023年9月までのROEを計算すると以下のように求められた.

\[
ROE = -\frac{680,386}{165,499} = -4.110
\]

\[
売上高利益率 = -\frac{680,386}{2,007,927} = -0.338
\]

\[
総資本回転率 = \frac{2,007,927}{2,098,296} = 0.957
\]

\[
財務レバレッジ率 = \frac{2,098,296}{165,499} = 12.682
\]

財務諸表より,それぞれ以下のように求められた.

\[
ROE = \frac{184,387}{694,879} = 0.265
\]

\[
売上高利益率 = \frac{184,387}{2,852,780} = 0.065
\]

\[
総資本回転率 = \frac{2,852,780}{2,415,319} = 1.181
\]

\[
財務レバレッジ率 = \frac{2,415,319}{694,879} = 3.475
\]

以上の結果から,ROEは2023年10月から2024年9月までに大幅に改善されたことがわかる.
財務レバレッジ率が大幅に減少し,売上高利益率,総資本回転率が増加したことから,初期費用が大きかったが,その後の売上高が増加し,収益性が向上したと考えられる.

\subsection{タイミーとの比較分析}
株式会社タイミーは,「一人ひとりの時間を豊かに」をミッションに,働きたい時間と働いてほしい時間をマッチングする「タイミー」というサービスと,地方で「はたらく」体験を通じて第二の故郷を見つけられる「タイミートラベル」というサービスを展開している.
2017年に設立され,2024年に東京証券取引所グロース市場に上場した.

まず,タイミーのROEを計算する.
2021年11月から2022年10月までのROEを計算すると以下のように求められた.

\[
ROE = \frac{256,751}{4,379,895} = 0.059
\]

\[
売上高利益率 = \frac{256,751}{6,216,517} = 0.041
\]

\[
総資本回転率 = \frac{6,216,517}{4,390,717} = 1.415
\]

\[
財務レバレッジ率 = \frac{4,390,717}{4,379,895} = 1.002
\]

2022年11月から2023年10月までのROEを計算すると以下のように求められた.

\[
ROE = \frac{1,802,769}{6,201,964} = 0.290
\]

\[
売上高利益率 = \frac{1,802,769}{16,144,584} = 0.112
\]

\[
総資本回転率 = \frac{16,144,584}{16,830,027} = 0.959
\]

\[
財務レバレッジ率 = \frac{16,830,027}{6,201,964} = 2.713
\]

SchooのROE分解とタイミーのROE分解を比較すると,以下のようになった.
\begin{table}[H]
  \begin{center}
    \caption{ROE分解比較}
    \begin{tabular}{|c|c|c|c|c|c|c|} \hline
      & ROE & 売上高利益率 & 総資本回転率 & 財務レバレッジ率 \\ \hline
      Schoo & 0.265 & 0.065 & 1.181 & 3.475 \\ \hline
      タイミー & 0.290 & 0.112 & 0.959 & 2.713 \\ \hline
    \end{tabular}
  \end{center}
\end{table}

Schoo株式会社は,売上高利益率が低く,総資本回転率が高く,財務レバレッジ率が高い.
一方,タイミーは,売上高利益率が高く,総資本回転率が低く,財務レバレッジ率が低い.

\subsection{戦略評価}
Schooは,前節の分析から,積極的にとうしを行い,効率的に売り上げをあげているものの,利益を十分に確保できていないことが考えられる.
決算説明資料より,事業戦略として「社会人教育の第一想起」を獲得することを目指し,国内社会人教育市場のNo.1を目指すとしている.
また,リカーリング収益モデルを採用しており,これは契約の継続の維持・向上が重要である.

一方,タイミーは,前節の分析から,高い利益率を確保できているものの,資産を効率的に活用できていなく,自己資本中心の経営を行っていると考えられる.
決算説明資料より,事業戦略としてサービスの拡充・開発を継続的に行い,ワーカーとクライアント双方のニーズに対応することを目指している.

以上の内容をもとにこれら2社の戦略の比較をしてみると,Schooは成長を重視した戦略を取っていると考えられる.
積極的に投資を行い,顧客基盤を拡大することを重視していると考えられる.

一方,タイミーは,収益性を重視した戦略を取っていると考えられる.
今後は,アライアンスやサービス拡充により事業の多角化を図ることでさらなる成長を目指すと考えられる.

\section{企業価値評価}
\subsection{DCFモデルによる評価}

\[
企業価値 = \frac{FCF_1}{1 + r} + \frac{FCF_2}{(1 + r)^2} + \cdots + \frac{FCF_n}{(1 + r)^n}
= \sum_{t=1}^{n} \frac{FCF_t}{(1 + r)^t}
\]

\[
割引率(WACC) r = \frac{自己資本比率 \times 株主資本コスト}{負債比率 \times 負債コスト \times (1 - 法人税)}
\]

\begin{table}[H]
  \begin{center}
    \caption{DCFモデルによる企業価値評価}
    \begin{tabular}{|c|c|c|c|c|c|c|c|} \hline
      年度 & 2024(当期) & 2025 & 2026 & 2027 & 2028 & 2029 & その後の成長 \\ \hline
      営業CF & 115,178 &  &  &  &  &  & \\ \hline
      投資CF & 19,806 &  &  &  &  &  & \\ \hline
      FCF & 95,372 & 133,520 & 186,929 & 243,007 & 315,910 & 410,683 & 431,217 \\ \hline
      FCFの成長率 &  & 40.0\% & 40.0\% & 30.0\% & 30.0\% & 30.0\% & 5.0\% \\ \hline
      株主資本コスト & 8${\%}$ & 8${\%}$ & 8${\%}$ & 8${\%}$ & 8${\%}$ & 8\% & 8\% \\ \hline
      負債コスト & 2.78\% & 2.78\% & 2.78\% & 2.78\% & 2.78\% & 2.78\% & 2.78\% \\ \hline
      自己資本比率 & 26.91\% & 26.91\% & 26.91\% & 26.91\% & 26.91\% & 26.91\% & 26.91\% \\ \hline
      WACC &  & 3.58 & 3.58 & 3.58 & 3.58 & 3.58 & 3.58 \\ \hline
      FCFの現在価値 &  & 128,905 & 172,230 & 218,670 & 274,447 & 361,673 & \\ \hline
    \end{tabular}
  \end{center}
\end{table}

\[
企業価値 = \sum_{t=1}^{5} {FCFの現在価値} + \frac{6年目のFCF}{WACC}
\]

\subsection{DDモデルによる評価}
% 配当割引モデルによる株価算定
\subsection{公開価格と初値の評価}
% 算定された理論株価と実際の株価との比較

\section{結論}
% 分析結果のまとめと今後の展望

\appendix
\section{財務データ}
% 詳細な財務データの表

\section{評価モデルの前提条件}
% DCFモデルとDDモデルの計算前提

\begin{thebibliography}{9}
\bibitem{reference1} 参考文献
\end{thebibliography}

\end{document}